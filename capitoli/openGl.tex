
\section{OpenGl}

\begin{figure}[hbp]
\centering
\includegraphics[width=0.7\textwidth]{images/openGL/OpenGL_logo.jpg}
\end{figure}


OpenGL (Open Graphics Library) è un'interfaccia che offre procedure per scrivere applicazioni di grafica 2D e 3D. Questa API è stata introdotta nel 1992 dalla società californiana Silicon Graphics, è nata in ambiente Unix e poi è stata resa multipiattaforma. OpenGL è usato per vari tipi di applicazioni, come videogiochi, realtà virtuale, animazioni e simulazioni 3D.

Ogni scheda grafica usa un linguaggio proprio, perciò, onde evitare problemi di compatibilità, le applicazioni dovrebbero disporre di più implementazioni, per essere eseguite sotto ogni tipo di acceleratore grafico. Il vantaggio di sfruttare API come OpenGL sta nel fatto che questa difficoltà è superata poichè offrono un'interfaccia che permette al programmatore di utilizzare un linguaggio unico per comunicare in modo rapido ed efficiente con l'hardware. Oramai quasi la totalità delle schede grafiche supporta OpenGL, perciò è possibile nascondere tutta la complessità di interfacciamento utilizzando questa API; in questo modo è possibile sfruttare appieno l'accelerazione grafica ed ottenere le massime prestazioni nel rendering delle scene.



\subsection{Il funzionamento di OpenGL}
OpenGL offre circa 250 chiamate di funzione per disegnare scene a due o tre dimensioni, a partire da delle primitive. Con primitive si intendono punti, linee e poligoni, i quali vengono convertiti in pixel tramite una serie di processi chiamata pipeline grafica.
La pipeline rappresenta una catena di trasformazioni e operazioni, in cui ogni fase prende in input il risultato della fase precedente e produce l'output per la fase successiva.

OpenGl opera a basso livello e richiede al programmatore di rispettare i passi precisi della pipeline che servono a renderizzare la scena. Tra i passi principali da seguire sono presenti :

\begin{itemize}
\item fornire le primitive che descrivano la scena.
\item fornire le regole per generare una telecamera virtuale, che renderizzi solo determinate porzione della scena, con particolari modalità.
\item fornire le regole per gestire texture, materiali, luci e ombre.
\end{itemize}

Queste regole vanno definite in particolari file chiamati "shader", scritti in un linguaggio ad alto livello proprio di OpenGL, basato sul linguaggio C, chiamato GLSL ("OpenGL Shading Language"). Siccome la scena non può essere renderizzata senza aver fornito le regole basilari, è necessaria, da parte del programmatore, una buona conoscenza della suddetta pipeline grafica e del linguaggio GLSL.

Esistono anche API che operano ad alto livello, le quali nascondono al programmatore le fasi più complesse del rendering, gestendole automaticamente, e richiedono solamente una descrizione generica della scena. Un'esempio è il framework Ogre3D, trattato in seguito.
