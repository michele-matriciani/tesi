
\section{OpenGl}

\begin{figure}[hbp]
\centering
\includegraphics[width=0.7\textwidth]{images/openGL/OpenGL_logo.jpg}
\end{figure}


OpenGL (Open Graphics Library) è una specifica che definisce un'API (“Application Programming Interface”, “interfaccia di programmazione per applicazioni”) che fornisce metodi per sviluppare applicazioni di grafica 2D e 3D \cite{opengl-wiki}. E' stata introdotta nel 1992 dalla società californiana Silicon Graphics, è nata in ambiente Unix e, successivamente, è stata resa multipiattaforma. La libreria OpenGL è usata per vari tipi di applicazioni, come animazioni, videogiochi, simulazioni 3D e di realtà virtuale.

Una specifica è un documento che descrive un insieme di funzioni ed il comportamento preciso che queste devono avere. I maggiori produttori hardware (Nvidia, AMD, Intel, etc) creano implementazioni di queste funzioni rispettando la specifica OpenGL, usando l'accelerazione hardware quando possibile. In questo modo il programmatore può usufruire di un'API unificata, utilizzando le funzioni senza preoccuparsi della loro implementazione.

Un'altra caratteristica fondamentale è che le funzioni di OpenGL devono essere sempre usufruibili, quindi le implementazioni richiedono un'emulazione software quando l'hardware non è in grado di fornire certe funzionalità.

\subsection{Il funzionamento di OpenGL}
OpenGL offre circa 250 chiamate di funzione per disegnare scene a due  o tre dimensioni, a partire da delle primitive. Con primitive si intendono punti, linee e poligoni, i quali vengono convertiti in pixel tramite una serie di processi chiamata pipeline grafica.
La pipeline rappresenta una catena di trasformazioni e operazioni, in cui ogni fase prende in input il risultato della fase precedente e produce l'output per la fase successiva.

OpenGL opera a basso livello e richiede al programmatore di rispettare i passi precisi della pipeline che servono a renderizzare la scena. Tra i passi principali da seguire sono presenti :

\begin{itemize}
\item fornire le primitive che descrivano la scena.
\item fornire le regole per generare una telecamera virtuale, che renderizzi solo determinate porzione della scena, con particolari modalità.
\item fornire le regole per gestire texture, materiali, luci e ombre.
\end{itemize}

Queste regole vanno definite in particolari file chiamati "shader", scritti in un linguaggio ad alto livello proprio di OpenGL, basato sul linguaggio C, chiamato GLSL ("OpenGL Shading Language"). Siccome la scena non può essere renderizzata senza aver fornito le regole basilari, è necessaria, da parte del programmatore, una buona conoscenza della suddetta pipeline grafica e del linguaggio GLSL.

Esistono anche framework che operano ad alto livello, i quali nascondono al programmatore le fasi più complesse del rendering, gestendole autonomamente, e richiedono solamente una descrizione generica della scena. Un'esempio è il framework Ogre3D, trattato in seguito.

\clearpage