
\chapter{Realtà Virtuale}
\fancyhead[RO]{\bfseries Realtà virtuale}

Al giorno d'oggi con \textbf{realtà virtuale} si intende quella disciplina che ha come obiettivo simulare la realtà, cioè permettere all'utente di interagire, in modo realistico e naturale, con un sistema puramente virtuale. A livello teorico, un sistema ideale dovrebbe permettere all'utente un'immersione completa, con la possibilità di utilizzare tutti i sensi. Le applicazioni che fanno parte di questa categoria sono ogni tipo di simulazione virtuale creata attraverso computer, schermi, dispositivi dotati di sensori, telecamere, etc. Tra queste, le più comuni sono i videogiochi, che stanno prendendo piede grazie all'introduzione del 3D e di periferiche di gioco sempre più immersive.
Se al giorno d'oggi con realtà virtuale si intende ogni forma di interazione con un sistema virtuale, nella storia essa ha avuto significati più ristretti.
 
\section{Storia} 
Il termine \textbf{“realtà virtuale”} fu introdotto nel 1980  per indicare l'insieme dei fenomeni percettivi indotti da un'apparecchiatura cibernetica a più componenti, che viene applicata a un soggetto umano\cite{realta2}.
Ad introdurre il termine fu un informatico statunitense che ha fondato la VPL Research\footnote{“Ricerca per i Linguaggi di Programmazione Virtuale”}, ovvero \textbf{Jaron Lanier}.\\
Tra i dispositivi creati in questo periodo possiamo citare  il \textbf{DataGlove}, un guantone-sensore collegabile ad un computer, inventato da Thomas Zimmermann, oppure l'\textbf{HMD} , “Display Head-Mounted”, un visore comprendente degli schermi e dei sensori per l'orientamento, ideato da  Scott Fisher, un ricercatore della Nasa.

Tuttavia, in passato, quando il concetto di realtà virtuale ancora non esisteva, già erano state costruite apparecchiature che possiamo considerare appartenenti a questa categoria.\\
Uno dei primi sistemi di realtà virtuale fu sviluppato nel 1962 da Morton Heilig, il quale costruì il prototipo di un dispositivo chiamato \textbf{Sensorama}, che proiettava cinque film e coinvolgeva sensi quali vista,olfatto,udito e tatto, creando, come la definiva lui, una sorta di \textit{“Cinema Esperienza”} (“Experience Theater”).
Nel 1977 al MIT venne creato l'\textbf{Aspen Movie Map}, una simulazione in cui l'utente poteva camminare ed esplorare la cittadina di Aspen, in Colorado, tramite una serie di filmati che inquadravano la maggior parte delle zone percorribili, nei limiti del possibile.

Al giorno d'oggi, uno dei dispositivi più efficienti per la realtà virtuale, tra quelli commercializzati, è l'\textbf{Oculus Rift}, che permette di vivere esperienze in prima persona grazie ad un visore da indossare, che possiede due schermi per avere una visione tridimensionale.
I commenti di coloro che hanno provato questo dispositivo sono positivi, in quanto la qualità della grafica e del sensore di movimento rendono l'esperienza quasi realistica.

Per ora ci si concentra prevalentemente sullo sviluppo di applicazioni e dispositivi destinati all senso della vista. Gli altri sensi ancora sono agli esordi della realtà virtuale e stanno prendendo piede molto lentamente. Questo probabilmente perché la vista è il senso più sviluppato nell'uomo, e quindi il più adatto a percepire il realismo delle simulazioni virtuali. 

%Grazie alle tecnologie di ultima generazione e al costante miglioramento della loro efficienza, gli effetti che si possono creare sono sempre più realistici.
%La grafica oramai è giunta a livelli di qualità elevatissimi, inoltre, grazie all'aggiunta dell'effetto 3D, i risultati ottenibili sono maggiormente in grado di far immergere l'utente all'interno del mondo virtuale.\\





