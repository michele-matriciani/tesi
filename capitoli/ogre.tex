
\section{Ogre3D}
\textbf{Ogre3D} (“Object-Oriented Graphics Rendering Engine”) è un motore grafico 3D open-source, scritto in C++, che essenzialmente offre un'API che sfrutta l'accelerazione hardware per la creazione di applicazioni grafiche. 

\begin{figure}
\centering
\includegraphics[width=0.6\textwidth]{images/ogre/ogrelogo-small.jpg}

\end{figure}

\subsection{Storia}
E' stato ideato nel 1999 da uno sviluppatore inglese di nome \textbf{Steve Streeting}, dopo che ebbe creato un progetto chiamato \textbf{DIMClass}, il quale aveva lo scopo di rendere più semplice l'utilizzo della libreria Direct3D. Si rese conto che il livello di astrazione raggiunto dalla sua libreria era cosi elevato che non aveva più bisogno di essere basata su Direct3D. Da qui nacque l'idea di creare una libreria che fosse indipendente da API a basso livello \cite{ogre-wiki}.

Nel 2000 il progetto fu registrato su \textbf{SourceForge}\footnote{ Piattaforma web per lo sviluppo di progetti e software, facilitando la collaborazione tra sviluppatori }, con il nome di OGRE. Il 25 febbraio 2005 fu rilasciata la  prima versione di Ogre, la 1.0.0, con il nome di “Azathoth”. La versione utilizzata in questo progetto è la 1.9.0, rilasciata il 24 novembre 2013, con il nome “Ghadamon”.

\subsection{La struttura ad albero}
Ogre3D utilizza una struttura dati ad albero (\textbf{scene graph}), dove ogni nodo può avere più figli ma un solo padre. Ogni nodo permette di gestire determinate caratteristiche della scena, e ogni suo comportamento si riflette in tutti i suoi discendenti. I principali nodi sono, a partire dalla radice e procedendo gerarchicamente:
\begin{itemize}
\item \textbf{Root}: permette di configurare il sistema.
\item \textbf{Scene Manager}: gestisce tutta la scena. Solitamente si usa un solo Scene Manager, ma è possibile crearne altri. Utilizzare più Scene Manager permette di renderizzare la scena in modo diverso; questo può essere usato ad esempio per suddividere la finestra in più parti, ognuna delle quali inquadra una zona diversa della scena (split screen).
\item \textbf{Scene Node}: appendendo diversi oggetti a questo nodo, è possibile creare gruppi di elementi, in modo da poter applicare effetti o trasformazioni all'intero gruppo, utilizzando una singola istruzione.
\item \textbf{Oggetti}: tra i singoli oggetti utilizzabili, i principali sono le telecamere (\textbf{Camera}), le luci (\textbf{Light}) e le entità (\textbf{Entity}). Queste ultime rappresentano i modelli 3D, che possono essere importati dall'esterno, o creati grazie a metodi che costruiscono primitive quali piani o cubi. 
\end{itemize}
%Ogre3D utilizza una struttura dati ad albero (\textbf{scene graph}), dove ogni nodo può avere più figli ma un solo padre. Partendo dalla radice, che permette di configurare il sistema, si possono poi creare dei figli che gestiscono la scena diversamente. Il nodo principale è lo \textbf{Scene Manager} che, come dice il nome, gestisce tutto ciò che è presente nella scena. Solitamente si utilizza un solo Scene Manager, ma è sempre possibile crearne di altri, magari in applicazioni più complesse, per gestire più scene. Questo può essere utilizzato ad esempio per suddividere la finestra in più parti, ognuna delle quali riprende una zona diversa della scena (split screen).
%
%Dallo Scene Manager è possibile appendere dei nodi chiamati \textbf{Scene Node}, ai quali, a loro volta, possono essere collegati elementi come entità (i modelli), luci, telecamere, etc. In questo modo è possibile raggruppare più elementi ed applicare all'intero gruppo effetti o trasformazioni con una singola istruzione.
%
%Infine troviamo i singoli oggetti. Tra questi vi sono i modelli, rappresentati da oggetti chiamati \textbf{Entity}, i quali devono essere imparentati ad almeno uno Scene Node, tramite il quale sono gestiti. E' possibile creare dei modelli preimpostati, come ad esempio cubi o piani, oppure si possono importare dall'esterno. Tra gli altri oggetti utilizzabili, i più comuni sono luci e telecamere.

\subsection{I vantaggi del framework}
Ogre3D è un framework che permette di astrarre tutta la parte che deve essere implementata in API a basso livello. Infatti, molte operazioni che devono essere esplicitate negli altri casi, qui sono gestite automaticamente: utilizzando solamente poche righe di codice è possibile richiamare metodi che implementano queste operazioni, nascondendole all'utente.
Un esempio è la creazione di modelli: Ogre3D offre dei metodi che costruiscono figure di base, oppure permettono di importare modelli sotto forma di file con estensione “.mesh”. Tutta la parte relativa al processamento dei vertici e al loro salvataggio nei buffer, viene nascosta.

Per quanto riguarda texture, materiali, luci ed ombre, Ogre3D contiene di default degli algoritmi di shading che li gestiscono. Anche la telecamera virtuale è gestita dal framework: all'utente viene richiesta solamente l'impostazione dei valori che la definiscono, mentre tutta la parte relativa alle trasformazioni viene nascosta.

Se da una parte l'autonomia del sistema porta a dei vantaggi, questo può determinare anche delle limitazioni. Fortunatamente Ogre3D lascia comunque ampie libertà: ad esempio l'utente può utilizzare materiali\footnote{I materiali sono descritti in file con estensione .material, tramite un linguaggio proprio di Ogre3D, di facile comprensione.} e shader personalizzati o preimpostati, forniti da Ogre3D al momento dell'installazione. Un altro esempio è la possibilità di definire la telecamera virtuale tramite matrici personalizzate. In particolare è possibile impostare la view matrix e la projection matrix, aspetto fondamentale ai fini della nostra applicazione. 
%
% All'utente viene comunque lasciata un'ampia libertà in quanto può utilizzare anche shader preimpostati, forniti da Ogre3D, oppure shader personalizzati, da includere nelle risorse del programma che si vuole creare.
%
%
%
%Inoltre in OpenGL la gestione di texture, ombre e materiali va implementata a mano scrivendo gli algoritmi di "shading", che contengono informazioni quali il colore o regole di comportamento in presenza di luce. Al contrario in Ogre la gestione di questi aspetti è quasi automatica in quanto di default vengono utilizzati shader già presenti nel framework.
%
%Tuttavia, Ogre è anche versatile perchè lascia all'utente ampie libertà, permettendogli di personalizzare diversi aspetti. Ad esempio in Ogre la view matrix e la perspective matrix, che servono a definire la telecamera, sono generate in automatico, in modo trasparente al programmatore. Poichè la nostra trasformazione richiedeva che le due matrici fossero in funzione della posizione dell'utente, esse sono state calcolate in una funzioni scritte a mano, e poi passate come parametro a due metodi che impostano le trasformazioni a partire da matrici personalizzate.
%
%Inoltre è sempre possibile utilizzare shader personalizzati, e definire i materiali degli oggetti in file con estensione .material, scritti con un linguaggio di facile comprensione, proprio di Ogre.
%
%Per questo, una volta compresa la trasformazione che richiedeva il nostro progetto, è stata abbandonata l'idea di utilizzare OpenGL passando invece a Ogre3D, con lo scopo di concentrarci sulla qualità e la miglior resa della scena.
