
\chapter*{Conclusioni}
In questa tesi è stata trattata un'applicazione che presenta un'interfaccia uomo-macchina per l'interazione con un sistema virtuale. In particolare rende lo stesso utente il controller del sistema visivo per una scena virtuale. La prospettiva con cui è visualizzata la scena è trasformata in funzione delle coordinate del volto rilevato. L'obiettivo era creare l'illusione della presenza di profondità all'interno dello schermo, generando un effetto tridimensionale.

L'applicazione è stata sviluppata utilizzando software completamente open-source. Sono stati usati:
OpenCV per il rilevamento del volto, OpenGL per lo studio di fattibilità e la produzione di un prototipo, Ogre3D per il miglioramento della resa della scena. Inoltre per la modellazione 3D si è usato Blender.

L'applicazione offre un'ulteriore feature, ovvero la possibilità di importare modelli dall'esterno. Esistono exporter che permettono di esportare scene, modellate con diversi software, nel formato riconosciuto da Ogre3D. Tuttavia per ora si è preso in considerazione solamente l'exporter di Blender, perciò il funzionamento è garantito solo per scene esportate con questo software.

Richiedendo solamente un pc e una webcam, il programma è accessibile da tutti, ma presenta dei limiti. Il limite che influisce maggiormente è l'assenza del calcolo preciso della distanza dell'utente dallo schermo. Questo problema rende non uniformi gli spostamenti dell'utente, se si trova a distanze differenti.

Al di là dei limiti, l'applicazione ha riscosso giudizi positivi, soprattutto da parte dell'utente medio, magari non abituato ad illusioni di questo genere, perciò il risultato si può considerare più che accettabile.

L'applicazione presenta grandi potenzialità, potendo estendere il suo utilizzo per svariati scopi, a partire da semplice forma di intrattenimento, fino ad essere impiegato in simulazioni virtuali più complesse, come ad esempio i videogiochi. Il fatto che l'applicazione sia stata sviluppata con software open-source la rende completamente aperta a miglioramenti e sviluppi futuri. 
